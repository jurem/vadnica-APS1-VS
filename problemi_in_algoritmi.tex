\chapter{Problemi in algoritmi}

\section{Osnovni pojmi}

\exe{Zakaj je \vic{mestni} (arabski oz. indijski) zapis števil tako pomemben (tudi za algoritmiko)? Namig: predstavljajte si algoritem za seštevanje (ali pa množenje) rimskih številk.}


\exe{Kaj je \vic{število} \angl{number}, \vic{številka} \angl{numeral} in \vic{števka} \angl{digit}? In kaj je \vic{cifra} in kaj \vic{mož}?}


\exe{Kaj je več $2^{10}$ ali $10^2$?}


\exe{Kaj je \vic{bit} in kaj je \vic{bajt}? Kaj je več 42 kB ali 42 KiB? Koliko bitov je v 42 MiB?}


\exe{Katere osnove navadno uporabljajo logaritmske funkcije v algoritmiki: $\log n$, $\lg n$ in $\ln n$?}
\ans{Dogovor je naslednji: $\log n=\log_{10} n$, $\lg n=\log_2 n$ in $\ln n=\log_e n$.}


\exe{Od kje oz. od koga pride izraz \vic{algoritem}? S kakšnimi algoritmi se je ukvarjala dotična oseba?}


\exe{Opredeli (intuitivno, vendar natančno) pojem \vic{algoritma}. Obrazloži pomembne dele definicije.}


\exe{Sošolki povej en \vic{dvoumen} in en \vic{nejasen} stavek (vendar ne oboje hkrati). Kaj je zahtevnejše: iskanje sošolke ali stavka?}


\exe{Kaj je \vic{računski problem}? Podaj primer računskega problema, ki ni povezan z računanjem.}


\exe{Pojasni razliko med \vic{problemom} (kadar v algoritmiki rečemo problem imamo v mislih računski problem), \vic{nalogo} in \vic{rešitvijo}.}


\exe{Naštej in obrazloži vrste \vic{računskih problemov}:
\begin{itemize}
\item iskalni,
\item odločitveni,
\item preštevalni,
\item naštevalni in
\item optimizacijski.
\end{itemize}
Za vsako vrsto podaj primer ali dva.}


\exe{Preveri veljavnost trditve:
\begin{enumerate}
\item Seštevanje, odštevanje, množenje dve števil so računski problemi, iskanje najmanjšega elementa v seznamu števil pa ni.
\item Za dana števila $x,y,z$ je vprašanje ali je $x+y=z$ odločitiveni problem.
\item Ali v danem seznamu elementov obstaja dani element je iskalni problem.
\item Urejanje seznama $5,2,9,3$ je računski problem.
\item Naloga problema poišči najbližjo točko koordinatnemu središču je seznam točk $(3,2), (1,5),(4,-2)$
\end{enumerate}
}


\exe{Zakaj je Turingov stroj pomemben za algoritmiko? Oglej si poljuben film o Alanu Turingu.}


\section{Snovanje in implementacija algoritmov}

\exe{Kaj je \vic{predpogoj} za dobro snovanje algoritmov?}
\ans{Dobro razumevanje problema preko natančne (matematične) definicije.}


\exe{Obrazloži nekaj kriterijev po katerih ocenjujemo kakovost algoritmov. Kateri kriterij je najpomembnejši?}
\ans{Pravilnost, učinkovitost, prilagodljivost, enostavnost, implementabilnost. Pravilnost je temelj vsakega algoritma.}


\exe{Naštej in primerjaj načine (\vic{opisni jeziki}) za opis algoritmov. Kateri načini so primernejši za ljudi in kateri za računalnike?}


\exe{Sošolki v naravnem jeziku obrazloži algoritem za iskanje največjega elementa v tabeli? Nato skupaj narišita diagram poteka za ta algoritem.}


\exe{Obrazloži faze razvoja algoritma od idejene zasnove do njegovega izvajanja. Obrazloži posamezne stopnje in semantične vrzeli med njimi. Kateri del je bolj abstrakten in kateri manj?}


\exe{Naštej nekaj pristopov oz. \vic{metod za snovanje} algoritmov. Več zabave s tem bo v sledečih poglavjih.}


\exe{Kaj je \vic{sintaktična} in kaj \vic{semantična} napaka v programu? Kaj je \vic{programski hrošč?}}


\exe{Naštej nekaj načinov za \vic{razhroščevanje} kode?}


\exe{Kaj je \vic{profiliranje} in kaj \vic{instrumentacija} kode?}


\exe{Kaj je \vic{sled} algoritma?}


\exe{Ali za izvajanje algoritma vedno potrebujemo računalnik? Obrazloži.}


\section{Algoritmi od vsepovsod}


\exetitlbl{Največji in najmanjši element}{alg:minmax}{Zasnuj algoritme za iskanje največjega in najmanjšega elementa ter oboje hkrati. Kateri izmed algoritmov naredi manj primerjav elementov?}


\exetitlbl{Zaporedno iskanje}{alg:seqsearch}{Zasnuj algoritem za iskanje danega elementa v dani tabeli. V čem je razlika v nalogi tega problema v primerjavi s problemom \quot{največji in najmanjši element} iz predhodne naloge?}


\exetit{Ugani število}{S sošolko igrajta igro \quot{ugani število}: zamisli si število med 1 in 128, ona pa naj ugiba, možni odgovori so manjše, enako, večje. Koliko ugibanj potrebuje v najslabšem primeru v različnih pristopih, npr. zaporedno iskanje, razpolavljanje (bisekcija).}


\exe{Načelo razpolavljanja (bisekcija) je eno izmed najbolj uporabnih načel v algoritmiki (in življenju nasploh). Kje se še uporablja?}
 

\exetitlbl{Dvojiško iskanje}{alg:binsearch}{Zasnuj algoritem \vic{dvojiško iskanje}, ki uporablja načelo razpolavljanja, za iskanje števila v urejenem zaporedju. V čem je razlika med nalogo tega problema in nalogo problema \quot{zaporedno iskanje} iz naloge \refexe{alg:seqsearch}? Zapiši tako rekurzivno kot iterativno obliko algoritma.}


\exetitlbl{Množenje s prištevanjem}{alg:mul-with-add}{Zasnuj algoritem za množenje dveh števil preko prištevanja. Namig: pomagaj si z definicijo množenja $a\cdot b = \underbrace{b + b + \dots + b}_{a-\text{krat}}$.}


\exe{Kdo je bil Evklid iz Aleksandrije? S čim se je še ukvarjal poleg algoritmov?}


\exe{Opiši \vic{Evklidov algoritem} za iskanje \vic{največjega skupnega delitelja}. Zapiši tako rekurzivno kot iterativno obliko algoritma.}


\exe{Opiši še en algoritem za iskanje \vic{največjega skupnega delitelja}, ki deluje preko faktorizacije števil.}


\exe{Prikaži sled Evklidovega algoritma za števili a) 123 in 456, b) 321 in 654 ter b) 59 in 61.}
\ans{a) \begin{tabular}{lllll}
\# & a & b & q & r \\
\hline
0 & 123 & 456 0 & 123 \\
1 & 456 & 123 & 3 & 87 \\
2 & 123 & 87 & 1 & 36 \\
3 & 87 & 36 & 2 & 15 \\
4 & 36 & 15 & 2 & 6 \\
5 & 15 & 6 & 2 & 3 \\
6 & 3 & 2 & 0 \\
7 & 3 & 0 \\
\end{tabular}}


\exe{Kaj se zgodi po prvem koraku Evklidovega algoritma, če je prvo število manjše od drugega?}


\exe{S pomočjo \vic{Eratostenovega sita} izračunaj praštevila manjša od $N=42$.}


\exetit{Faktoriela}{Zapiši rekurzivni algoritem za izračun faktoriele glede na formulo $n!=n\cdot(n-1)!$ in $0!=1$. Ali zapisani algoritem vsebuje repno rekurzijo? Če ne, ga spremeni, da jo bo, nato pa vse skupaj spremeni v iteracijo. Opazuj spremembe!}
\begin{Answer}
Ne-repna rekurzija: \code{fac}, repna rekurzija: \code{factail} \\
\begin{minipage}{8cm}
\begin{rox}
fun fac(n) is
	if n == 0 then 1 else n * fac(n - 1)
\end{rox}
\end{minipage}
\begin{minipage}{8cm}
\begin{rox}
fun factail(r, n) is
	if n == 0 then r else factail(r * n, n - 1)
\end{rox}
\end{minipage}
\end{Answer}


\section{Preverjanje pravilnosti}

\exe{Na svetovnem spletu poišči nekaj primerov znanih programskih hroščev.}


\exe{Kaj je poglavitno vprašanje (intuitivno), ki si ga postavimo, ko preverjamo pravilnost nekega algoritma?}
\ans{Ali program deluje, tako kot mislimo, da bi moral delovati?}


\exe{Naštej (štiri) načine s katerimi lahko preverjamo pravilnost algoritmov.}


\exe{Utemelji pravilnost algoritma \quot{množenje s prištevanjem} (glej nalogo \refexe{alg:mul-with-add}) preko \vic{intuitivnega razumevanja}. Ali algoritem deluje za negativna števila?}


\exe{Zakaj se pri razvoju programov zelo pogosto uporablja testiranje s testnimi primeri? Zakaj za testiranje algoritmov to pogosto ni zadostno?}


\exe{Koliko različnih vhodov je možnih za Evklidov algoritem, če privzamemo 32-bitna števila? Koliko let bi trajalo popolno testiranje algoritma, če imamo na voljo testni sistem, ki vsako sekundo preizkusi miljardo ($10^9$) vhodov?}
\ans{Št. vhodov: $2^{64}\approx1,8\cdot 10^{19}$, čas testiranja: $2^{64} / 10^9 / 60 / 60 / 24 / 365 = 585$ let.}


\exe{Algoritem za kvadriranje kvadratnih matrik velikosti $n \times n$ želimo \vic{popolno testirati} s testnimi primeri. Če so vsi elementi matrik 8 bitna števila, zapišite (v odvisnosti od $n$)
\begin{enumerate}
	\item število različnih vhodov in
	\item število različnih vhodov, če namesto kvadriranja vzamemo potenciranje.
\end{enumerate}}
\ans{a) $256^{n^2}$ - za vsak element v matriki je $2^8$ različnih števil, vseh elementov je $n^2$, b) $256^{n^2+1}$ poleg matrike je vhodni podatke tudi potenca}


\exe{Dano je polje dolžine 200. Tina Sredinec je implementirala algoritem, ki izračuna sredinsko pozicijo $m$ v polju med pozicijama $l$ (leva meja) in $r$ (desna meja), po formuli $m = (l+r)/2$. Vse spremenljivke vsebujejo 8 bitna nepredznačena števila. Kje se skriva programski hrošč? Kako bi program popravil?}
\ans{Za $l=150$ in $r=190$ dobimo $l+r=340~(\bmod~256)=84$ in torej $m=(l+r)/2=42$, kar očitno ni sredina med 150 in 180. Popravek: $m=l+(r-l)/2$.}


\exe{Ugotoviti želimo pravilnost nekega algoritma za urejanje seznama. Kateri dve lastnosti moramo preveriti?}
\ans{Da rezultat vsebuje enake elemente kot vhodno zaporedje in da je rezultat urejen seznam.}


\exe{V čem je prednost \vic{formalnega dokazovanja pravilnosti} algoritmov. Na katerem matematičnem načelu sloni dokazovanje pravilnosti algoritmov, ki vsebujejo zanke?}
\ans{Zanesljivost pravilnosti, indukcija.}


\exe{Formalni dokaz pravilnosti algoritma pogosto temelji na indukciji. Kaj je \vic{matematična indukcija}, \vic{hipoteza}, \vic{osnovni primer}, \vic{induktivna predpostavka} in \vic{induktivni korak}? V algoritmiki pa za dokazovanje zank uporabljamo tudi \vic{zančne invariante}.}


\exe{S pomočjo matematične indukcije dokaži $\sum_{i=0}^{n}i=\frac{n(n+1)}{2}$.}


\noindent\begin{minipage}{7.5cm}
\exe{S pomočjo indukcije dokaži pravilnost algoritma za iskanje maksimuma v tabeli števil.}
\end{minipage}
\hspace{1em}
\begin{minipage}{7cm}
\vspace{1.2em}
\begin{rox}
m = a[0]
for i = 1 to n-1 do
	if a[i] > m then m = a[i]
\end{rox}
\end{minipage}


\exe{S pomočjo indukcije dokaži pravilnost \quot{množenja s prištevanjem}.}


\exe{Dokaži pravilnost Evklidovega algoritma. Uporabite znani izrek v zvezi s tem.}


\section{Namigi in rešitve izbranih nalog}

\shipoutAnswer

