\chapter{Zahtevnost algoritmov}

\section{Splošna vprašanja}

\exe{Kaj je zahtevnost algoritma?}
\ans{Zahtevnost algoritma pove katere in koliko virov potrebuje algoritem za svoje izvajanje (v nekem modelu računanja).}


\exe{Naštej nekaj virov, ki jih algoritem lahko potrebuje za svoje izvajanje.}


\exe{Kateri viri ustrezajo meri \vic{časa} in kateri \vic{prostora}?}


\exe{Kaj je \vic{Von Neumannov} model računalniške arhitekture?}


\exe{Kaj je RAM model računanja? Zakaj ga uporabljamo v algoritmiki?}


\exe{Kaj je natančna zahtevnost in kaj asiptotična zahtevnost algoritma?}


\exe{Od česa je lahko odvisna zahtevnost algoritma? Prikaži s primerom.}
\ans{Od algoritma, modela računanja in od velikosti vhoda in samih podatkov v vhodu.}


\exe{Izberi nek problem, nato naštej nekaj primerov nalog zanj, katerih težavnost je različna:
\begin{enumerate}
\item glede na velikost naloge
\item glede na podatke v sami nalogi.
\end{enumerate} }


\exe{Glede na (vhodne) podatke, katere vrste določanja zahtevnosti poznamo?}


\exe{Zakaj najpogosteje uporabljamo zahtevnost v najslabšem primeru?}


\exe{Kdo je bil John von Neumann? S čim vsem se je ukvarjal? Katere izmed algoritmov za urejanje je napravil?}


\section{Natančna zahtevnost}


\exe{Koliko korakov zahteva algoritem \quot{množenje s prištevanjem} (glej nalogo \ref{alg:mul-with-add})?}


\exe{Koliko korakov zahteva Eratostenovo sito za poljuben $N$? En korak je mišljen kot odstranjevanje večkratnikov nekega števila $X$.}
\ans{Odstranjevanje večkratnikov $X$ je potrebno do: hitro vidimo, da za $X<N$ ali tudi $X<N/2$. Z malo razmislega pa pridemo do $X\leq\sqrt{N}$.}


\noindent\begin{minipage}{7.5cm}
\exe{Določi natančno zahtevnost v številu primerjav elementov glede na najboljši, najslabši in povprečni primer za algoritem (zaporednega iskanje, glej \refexe{alg:seqsearch}). Pri tem je $n$ velikost polja $a$.}
\end{minipage}
\hspace{1em}
\begin{minipage}{7cm}
\vspace{1.2em}
\begin{lstlisting}
for i = 0 to n-1 do
	if a[i] == key then return i
return -1
\end{lstlisting}
\end{minipage}
\ans{Najboljši primer: 1, najslabši primer: $n$ in povprečni primer $(n+1)/2$.}


\exe{Zakaj se kot pomembna operacija, s katero ocenjujemo zahtevnost, pogosto pojavi primerjava elementov, primerjavo indeksov pa navadno zanemarimo?}


\exe{Določi natančno zahtevnost v smislu realnega časa na poljubnem RAM modelu računanja za algoritem \quot{zaporedno iskanje} iz naloge \refexe{alg:seqsearch}}
\ans{Najboljši primer: $c_1+c_2+c_3$, najslabši primer: $c_1\cdot(n+1)+c_2\cdot n+c_3$, povprečni primer: $\frac{c_1+c_2}{2}n+\frac{c_1+c_2}{2}+c_3$, kjer je $c_1$ zahtevnost preverjanja pogoja v odločitvenem stavku, $c_2$ cena primerjave elementov in $c_3$ cena stavka return.}


\exe{Kolikšna je globina rekurzije pri algoritmu \alg{dvojiškega iskanja}?}


\section{Asimptotična zahtevnost}

\section{Namigi in rešitve izbranih nalog}

\shipoutAnswer
